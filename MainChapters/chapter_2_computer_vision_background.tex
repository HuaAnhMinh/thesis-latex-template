\chapter{Lý Thuyết Về Thị Giác Máy Tính}
\label{computer_vision_background}

The quick brown fox jumps over the lazy dog

% Hiển thị code từ file
\lstinputlisting[language=C++]{SourceCode/hello.cpp}

\begin{lstlisting}[language=Python]
print("Hello, World!!!")
\end{lstlisting}

\begin{lstlisting}[language=JavaScript]
console.log('Hello, World!!!');
\end{lstlisting}

% Sử dụng block \begin{figure} để LaTex tự động thêm reference cho ảnh đến trang danh sách hình ảnh.
% [h]: tham khảo thêm tại https://www.overleaf.com/learn/latex/Inserting_Images#Placement
\begin{figure}[h]
	\centering
	\includegraphics[scale=0.75]{logo-khtn}
	\caption{Logo của Trường Đại học Khoa Học Tự Nhiên}
\end{figure}

% Sử dụng block \begin{table} để LaTex tự động thêm reference cho bảng đến trang danh sách bảng.
\begin{table}[h!]
	\centering
	\begin{tabular}{||c c c c||} 
		\hline
		Col1 & Col2 & Col2 & Col3 \\ [0.5ex] 
		\hline\hline
		1 & 6 & 87837 & 787 \\ 
		2 & 7 & 78 & 5415 \\
		3 & 545 & 778 & 7507 \\
		4 & 545 & 18744 & 7560 \\
		5 & 88 & 788 & 6344 \\ [1ex] 
		\hline
	\end{tabular}
	\caption{Table to test captions and labels}
	\label{table:1}
\end{table}

Tham khảo mẫu: \cite{1984-TeX-Knuth}