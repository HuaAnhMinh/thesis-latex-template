% Template KLTN cho SV trường ĐHKHTN
% Liên hệ: huaanhminh0412@gmail.com
% Last update: 19/03/2021

\documentclass[oneside,a4paper,13pt]{extreport}

% Font size 13pt trong Word và trong LaTex khác nhau về mặt kích thước (LaTex nhỏ hơn) nên sử dụng 2 lệnh dưới đây giúp tăng font size lên
\usepackage{scrextend}
\changefontsizes{13pt}

% Font tiếng Việt
\usepackage[T5]{fontenc}
\usepackage[utf8]{inputenc}
\DeclareTextSymbolDefault{\DH}{T1}

% Chỉnh định dạng và style title các phần
\usepackage{tocloft,calc}
\usepackage[explicit]{titlesec}

% Format số trang
\usepackage{fancyhdr}
\usepackage{lastpage}

% Tài liệu tham khảo
\usepackage[
	sorting=nty,
	backend=bibtex,
	defernumbers=true]{biblatex}
\usepackage[unicode]{hyperref} % Bookmark tiếng Việt
\addbibresource{References/references.bib}
\usepackage{blindtext}

\makeatletter
\def\blx@maxline{77}
\makeatother

% Chèn hình, các hình trong luận văn được để trong thư mục Images/
\usepackage{graphicx}
\graphicspath{ {Images/} }

% Package quản lý mã nguồn
\usepackage{listings}
\usepackage{color}

% Cái package listings ngu học trong LaTex không support cho JS nên đó là lý do có một mớ này:
\lstdefinelanguage{JavaScript}{
	keywords={break, case, catch, continue, debugger, default, delete, do, else, finally, for, function, if, in, instanceof, new, return, switch, this, throw, try, typeof, var, void, while, with},
	morecomment=[l]{//},
	morecomment=[s]{/*}{*/},
	morestring=[b]',
	morestring=[b]",
	sensitive=true
}

% Chèn và định dạng mã nguồn
\definecolor{codegreen}{rgb}{0,0.6,0}
\definecolor{codegray}{rgb}{0.5,0.5,0.5}
\definecolor{codepurple}{rgb}{0.58,0,0.82}
\definecolor{backcolour}{rgb}{0.95,0.95,0.92}
\lstdefinestyle{mystyle}{
    backgroundcolor=\color{backcolour},   
    commentstyle=\color{codegreen},
    keywordstyle=\color{magenta},
    numberstyle=\tiny\color{codegray},
    stringstyle=\color{codepurple},
    basicstyle=\footnotesize,
    breakatwhitespace=false,         
    breaklines=true,                 
    captionpos=b,                    
    keepspaces=true,                 
    numbers=left,                    
    numbersep=5pt,                  
    showspaces=false,                
    showstringspaces=false,
    showtabs=false,                  
    tabsize=4
}
\lstset{style=mystyle}

% Chèn và định dạng mã giả
\usepackage{amsmath}
\usepackage{algorithm}
\usepackage[noend]{algpseudocode}
\makeatletter
\def\BState{\State\hskip-\ALG@thistlm}
\makeatother

% Bảng biểu
\usepackage{multirow}
\usepackage{array}
\newcolumntype{L}[1]{>{\raggedright\let\newline\\\arraybackslash\hspace{0pt}}m{#1}}
\newcolumntype{C}[1]{>{\centering\let\newline\\\arraybackslash\hspace{0pt}}m{#1}}
\newcolumntype{R}[1]{>{\raggedleft\let\newline\\\arraybackslash\hspace{0pt}}m{#1}}

% Đổi tên mặc định
% Chapter -> Chương
% Figure -> Hình
% Table -> Bảng
% Table of Contents -> Mục lục (1)
% List of Figures -> Danh sách hình (2)
% List of Tables -> Danh sách bảng (3)
% Appendix -> Phụ lục (4)
% Note: Hiện tại các title ở các trang (1), (2), (3), (4) được định dạng kích thước Large (nhỏ hơn so với mặc định), và được canh giữa bằng 2 command \hfill đầu và cuối
\renewcommand{\chaptername}{Chương}
\renewcommand{\figurename}{Hình}
\renewcommand{\tablename}{Bảng}
\renewcommand{\contentsname}{\hfill\Large\centering Mục lục\hfill}
\renewcommand{\listfigurename}{\hfill\Large\centering Danh sách hình\hfill}
\renewcommand{\listtablename}{\hfill\Large\centering Danh sách bảng\hfill}
\renewcommand{\appendixname}{\hfill\Large\centering Phụ lục\hfill}
\renewcommand{\appendixname}{\hfill\Large\centering Phụ lục\hfill}
\renewcommand{\bibliography}{\hfill\Large\centering Tham khảo\hfill}
\renewcommand{\cftaftertoctitle}{\hfill}

% Dãn dòng 1.5 (line spacing = 1.5)
\usepackage{setspace}
\onehalfspacing

% Thụt vào đầu dòng
\usepackage{indentfirst}

% Canh lề (trên, dưới, trái, phải)
% Nếu định dạng canh lề trên dưới trái phải khác thì hãy chỉnh thông qua 4 biến bên dưới đây.
\newcommand{\topMargin}{30mm}
\newcommand{\bottomMargin}{35mm}
\newcommand{\leftMargin}{35mm}
\newcommand{\rightMargin}{20mm}

\usepackage[
  top=\topMargin,
  bottom=\bottomMargin,
  left=\leftMargin,
  right=\rightMargin,
  includefoot]{geometry}
  
% Trang bìa
\usepackage{tikz}
\usetikzlibrary{calc}
\newcommand\HRule{\rule{\textwidth}{1pt}}

\usepackage{pst-solides3d}

% Tên các sinh viên viết thường, nếu có nhiều hơn 2 sinh viên thì sử dụng \newcommand{\<tên_biến>}{tên~sv} để định dạng
% Note: dấu ~ trong tên khi build ra sẽ thành khoảng trắng. Ví dụ: Hứa~Anh~Minh -> Hứa Anh Minh
\newcommand{\firstStudent}{Hứa~Anh~Minh}
\newcommand{\secondStudent}{Nguyễn~Ngọc~Đăng~Khanh}

% Tên các sinh viên viết in hoa toàn bộ
\newcommand{\firstStudentUppercase}{HỨA~ANH~MINH}
\newcommand{\secondStudentUppercase}{NGUYỄN~NGỌC~ĐĂNG~KHANH}

% Mã số sinh viên
\newcommand{\firstStudentId}{1753070}
\newcommand{\secondStudentId}{1753134}

% Tên khóa luận / đồ án tốt nghiệp
\newcommand{\thesisNameUppercase}{XÂY~DỰNG~HỆ~THỐNG~PHÂN~TÍCH~\\SỰ~DI~CHUYỂN~CỦA~KHÁCH~HÀNG\\BẰNG~THỊ~GIÁC~MÁY~TÍNH}
\newcommand{\thesisNameLowercase}{Xây~Dựng~Hệ~Thống~Phân~Tích~Sự~Di~chuyển~Của~Khách~Hàng~Bằng~Thị~Giác~Máy~Tính}	

% Tên giáo viên hướng dẫn, nếu nhiều hơn 1 giáo viên hướng dẫn thì sử dụng lệnh \newcommand để tạo thêm biến cho giáo viên khác.
\newcommand{\tutorName}{Tiến~sĩ~Đinh~Bá~Tiến}
\newcommand{\tutorNameUppercase}{TIẾN~SĨ~ĐINH~BÁ~TIẾN}

\newcommand{\facultyName}{Khoa~Học~Máy~Tính~-~Công~Nghệ~Phần~<Mềm}

% Ngày xuất bản chính thức, thay ... bằng ngày tháng năm để hiển thị cho cả báo cáo
\newcommand{\releaseDate}{ngày~...~tháng~...~năm~...}


\renewcommand{\cftchappresnum}{Chương }
\AtBeginDocument{\addtolength\cftchapnumwidth{\widthof{\bfseries Chương }}}

% Custom level các section trong bookmark khi xem file PDF.
\usepackage[atend]{bookmark}
\bookmarksetup{
	numbered,
	addtohook={%
		\ifnum\bookmarkget{level}>1 %
		\bookmarksetup{numbered=false}%
		\fi
	},
}

% Custom lại title các section ở đầu trang thành canh giữa, kích thước Large, khoảng cách dòng giữa chữ Chương xx và tiêu đề là 10pt.
\titleformat{\chapter}[display] {\Large\bfseries\centering}{\filcenter\chaptertitlename\ \thechapter}
{10pt}{\Large\centering{#1}}

% Thêm dấu - vào giữa Chương xx và tên section trong mục lục.
% Mặc định: Chương 03 abcxyz
% Custom thành: Chương 03 - abcxyz
\renewcommand{\cftsecaftersnum}{~-~}
\renewcommand{\cftchapaftersnum}{~-~}
\renewcommand{\cftsubsecaftersnum}{~-~}

\begin{document}

% Front matter - những trang đầu

% Trang bìa
\pagestyle{empty}
\begin{titlepage}

% TRANG BÌA NGOÀI

\begin{center}
%ĐẠI HỌC QUỐC GIA THÀNH PHỐ HỒ CHÍ MINH\\
TRƯỜNG ĐẠI HỌC KHOA HỌC TỰ NHIÊN\\
\textbf{KHOA CÔNG NGHỆ THÔNG TIN}\\
\textbf{Chương Trình Chất Lượng Cao}\\[2cm]


{ \bfseries \firstStudentUppercase~-~\secondStudentUppercase\\[2cm] } 


{ \Large \bfseries \thesisNameUppercase \\[3cm] }


{ \bfseries KHÓA LUẬN TỐT NGHIỆP\\CHƯƠNG TRÌNH CHẤT LƯỢNG CAO\\[2cm] }

% Vẽ khung
\begin{tikzpicture}[remember picture, overlay]
  \draw[line width = 1pt] ($(current page.north west) + (\leftMargin, -\topMargin)$) rectangle ($(current page.south east) + (-\rightMargin,\bottomMargin)$);
\end{tikzpicture}

\vfill
Tp. Hồ Chí Minh, \releaseDate

\end{center}

\pagebreak

% TRANG BÌA PHỤ BÊN TRONG

\begin{center}

TRƯỜNG ĐẠI HỌC KHOA HỌC TỰ NHIÊN\\
\textbf{KHOA CÔNG NGHỆ THÔNG TIN}\\
\textbf{Chương Trình Chất Lượng Cao}\\[2cm]


{ \bfseries \firstStudentUppercase~-~\firstStudentId\\[0.5cm] } % [0.5cm]: khoảng cách giữa 2 dòng của 2 tên sv là 0.5cm
{ \bfseries \secondStudentUppercase~-~\secondStudentId\\[2cm] }


%Tên đề tài Khóa luận tốt nghiệp/Đồ án tốt nghiệp

{ \Large \bfseries \thesisNameUppercase \\[2cm] }


{ \bfseries KHÓA LUẬN TỐT NGHIỆP\\CHƯƠNG TRÌNH CHẤT LƯỢNG CAO\\[2cm] }

{ \bfseries NGƯỜI HƯỚNG DẪN\\\tutorNameUppercase\\ }


% Vẽ khung
\begin{tikzpicture}[remember picture, overlay]
	\draw[line width = 1pt] ($(current page.north west) + (\leftMargin, -\topMargin)$) rectangle ($(current page.south east) + (-\rightMargin,\bottomMargin)$);
\end{tikzpicture}

\vfill
Tp. Hồ Chí Minh, \releaseDate

\end{center}

\end{titlepage}

% Đánh số la mã cho các trang ở phần front matter
\pagenumbering{roman} % Đánh số i, ii, iii, ...
\pagestyle{plain}

% Trang dành cho người hướng dẫn và hội đồng viết nhận xét
\input{FrontMatter/comment-advisor.tex}
\newpage
\chapter*{Nhận xét của người đánh giá}

\begin{tikzpicture}[remember picture, overlay]
	\draw[line width = 1pt] ($(current page.north west) + (\leftMargin, -\topMargin)$) rectangle ($(current page.south east) + (-\rightMargin,\bottomMargin + 10mm)$);
\end{tikzpicture}

\multido{}{11}{\noindent\makebox[\linewidth]{\dotfill}\endgraf}

\indent Thành phố Hồ Chí Minh, \releaseDate \\
\indent Người đánh giá\\
\newpage

% Trong front matter hiện tại sẽ gồm các phần sau:
% Lời cảm ơn
% Mô tả
% Mục lục *
% Danh sách hình *
% Danh sách bảng *
% Đề cương chi tiết
% Note: các file tex cho các trang này (trừ những trang có *) sẽ được đặt trong thư mục FrontMatter
% Những trang có * là trang được config và thêm tự động vào bởi LaTex nên sẽ không cần thêm những trang đó.

% command \phantomsection để hyperlink trong mục lục trỏ đúng section
\phantomsection
\addcontentsline{toc}{chapter}{\mdseries Lời cảm ơn}
\chapter*{Lời cảm ơn}
\label{thanks}

Lời đầu tiên, chúng em xin cảm ơn sự giúp đỡ của thầy cô, bạn bè đã hỗ trợ và giúp đỡ chúng em về mặt vật chất và tinh thần để hoàn thành khóa luận này một cách tốt đẹp. \\

Chúng em xin gửi lời biết ơn sâu sắc đến thầy Đinh Bá Tiến đã dành thời gian giúp đỡ chúng em về mặt kiến thức, công cụ và gợi ý cho chúng em có thể thực hiện được khóa luận này. \\

Ngoài ra, chúng em xin cảm ơn anh Dương Nguyễn Thái Bảo đã hỗ trợ chúng em về mặt ý tưởng cũng như cung cấp các biểu mẫu các tài liệu cần thiết trong quá trình thực hiện luận văn này. \\

Cuối cùng, chúng em xin gửi lời cảm ơn đến những người thân trong gia đình vì tình yêu thương và sự động viên vô điều kiện của họ đã có tác dụng thúc đẩy chúng em rất nhiều để hoàn thành khóa luận này. \\ [2cm]

\begin{flushright}
	Thành phố Hồ Chí Minh, \releaseDate \\
	\firstStudent~-~\secondStudent
\end{flushright}

\phantomsection
\addcontentsline{toc}{chapter}{\mdseries Mô tả}
\chapter*{Mô tả}
\label{syllabus}

\cleardoublepage
\phantomsection
\addcontentsline{toc}{chapter}{\mdseries Mục lục}
\tableofcontents
\newpage

\cleardoublepage
\phantomsection
\addcontentsline{toc}{chapter}{\mdseries Danh sách hình}
\listoffigures
\newpage

\cleardoublepage
\phantomsection
\addcontentsline{toc}{chapter}{\mdseries Danh sách bảng}
\listoftables
\newpage

\cleardoublepage
\phantomsection
\addcontentsline{toc}{chapter}{\mdseries Đề cương chi tiết}
\chapter*{Đề cương chi tiết}
\label{abstract}

\clearpage

% Main chapters - những phần nội dung chính

% Những phần nội dung chính sẽ được đánh số 1, 2, 3,...

\pagenumbering{arabic} % Đánh số 1, 2, 3,...
\pagestyle{plain}

\chapter{Giới thiệu}
\label{introduction}

\section{Some dummy section}

% the paragraphs below are just dummy text. Replace it with your contents

Lorem ipsum dolor sit amet, consectetur adipiscing elit. Vivamus sed gravida ligula. Donec nisi leo, euismod vel aliquet quis, ultrices sed velit. Fusce at sodales massa. Interdum et malesuada fames ac ante ipsum primis in faucibus. Ut quis scelerisque arcu. Integer sit amet risus dui. Praesent ultrices diam eget libero eleifend molestie.

Proin congue neque id ipsum mollis, venenatis facilisis felis congue. Vivamus volutpat dapibus hendrerit. Nullam euismod, nisi a feugiat pretium, erat sem tempor tellus, ut laoreet tellus nisi ac ante. Phasellus tristique nunc semper, vehicula nulla quis, vulputate elit. Praesent fringilla lorem eget nulla scelerisque blandit. Suspendisse eu purus elit. Pellentesque efficitur eros libero, euismod condimentum libero congue ut. Cras quis massa cursus, bibendum dui et, finibus felis. Pellentesque habitant morbi tristique senectus et netus et malesuada fames ac turpis egestas.

Proin augue ipsum, cursus et nisl non, tempus dapibus diam. Curabitur venenatis lectus orci, eget egestas quam pulvinar eu. Duis aliquet, felis quis condimentum dignissim, lorem augue posuere elit, in tempor felis enim vitae magna. Vestibulum in rhoncus lacus. Aenean orci massa, volutpat et blandit at, fermentum imperdiet augue. Aenean et justo tincidunt, euismod ligula et, cursus libero. Pellentesque vel est pellentesque, porttitor nisi ac, rutrum diam.

Mauris euismod vulputate fermentum. Mauris lobortis augue arcu, congue pulvinar libero fermentum a. Donec sed tortor et urna porttitor condimentum at sed leo. Morbi mollis, neque vel ullamcorper posuere, nisl sem pellentesque arcu, non consequat erat ipsum iaculis lectus. Vestibulum porttitor risus diam, vel dapibus eros placerat vel. Fusce id placerat tortor. Suspendisse nec commodo arcu. Nunc nisi sapien, porttitor non nulla nec, tempor sollicitudin turpis. Nulla in luctus elit. Duis interdum dui massa, quis fringilla odio scelerisque vel. Aenean suscipit erat nec mattis dignissim. Pellentesque habitant morbi tristique senectus et netus et malesuada fames ac turpis egestas. Integer ut quam mi. Sed feugiat enim non ex feugiat vulputate. Quisque a dui eget mauris rhoncus venenatis. Mauris laoreet aliquet lectus, non semper turpis semper vel.

Pellentesque bibendum neque a dolor pulvinar bibendum in sed nisl. Donec tempus est libero, non consectetur mi mattis ut. In sit amet sapien in nibh fermentum sagittis. Donec a elit sollicitudin, venenatis lacus in, viverra lectus. Aenean ut velit a erat placerat fermentum sodales nec massa. Vestibulum ac nibh nec velit tincidunt pellentesque. Integer eleifend et lectus a lobortis. Duis sollicitudin tortor eu lacus auctor volutpat eget non sapien. Fusce et maximus est. Morbi vehicula ultricies aliquam. Pellentesque sit amet mauris dui.

\subsection{Some dummy sub section}

Lorem ipsum dolor sit amet, consectetur adipiscing elit. Vivamus vehicula ante nec pretium tempor. Curabitur pretium augue quis pellentesque congue. Nulla gravida eu libero a condimentum. Phasellus ullamcorper dolor id turpis tempor semper. Morbi eros eros, tempus id tempus eget, placerat id turpis. Aenean non ex in diam sagittis facilisis. Aenean accumsan massa tortor, ac efficitur diam mollis ut. Donec a diam a sapien pulvinar aliquam.

\section{Some dummy section 2}

Lorem ipsum dolor sit amet, consectetur adipiscing elit. Vivamus vehicula ante nec pretium tempor. Curabitur pretium augue quis pellentesque congue. Nulla gravida eu libero a condimentum. Phasellus ullamcorper dolor id turpis tempor semper. Morbi eros eros, tempus id tempus eget, placerat id turpis. Aenean non ex in diam sagittis facilisis. Aenean accumsan massa tortor, ac efficitur diam mollis ut. Donec a diam a sapien pulvinar aliquam.

Morbi at felis id lectus ornare mollis vel eu nulla. Duis velit risus, volutpat nec lorem a, imperdiet cursus risus. Duis aliquam sem magna, nec mattis tellus faucibus at. Quisque eu erat est. In metus nulla, hendrerit ut libero sed, commodo porttitor enim. Aliquam auctor nec erat non imperdiet. Etiam vel facilisis ligula. In fermentum, nisl in rhoncus molestie, velit tortor tincidunt risus, eu convallis lorem nisl fermentum nunc. Vivamus quis arcu at velit tempus porta eu id erat. Cras at dui egestas, fermentum diam sed, ultrices turpis. Sed dapibus maximus tellus ut condimentum. Cras in nisi in tortor accumsan imperdiet vel ut sem. Mauris consectetur venenatis ipsum eu faucibus. Donec viverra pretium interdum.

Donec hendrerit nulla eget nunc tempor ullamcorper. Fusce sed dolor dictum, ultrices nibh sed, ultrices enim. Nunc dignissim lobortis ipsum, ullamcorper vehicula est sodales non. Mauris ac risus libero. Vivamus dignissim orci eu ligula venenatis, sed faucibus dui dapibus. Nullam at ultricies sapien. Nullam sollicitudin volutpat purus, porta feugiat mi cursus id. Interdum et malesuada fames ac ante ipsum primis in faucibus. Quisque mollis sapien quam.

Suspendisse nec porttitor mauris, a fermentum enim. Phasellus vulputate pretium ante in dignissim. Mauris sit amet fringilla risus, ac egestas ipsum. Donec dui tortor, tristique id lacinia eget, porttitor in risus. In varius convallis justo. Pellentesque tincidunt aliquet lacus quis facilisis. Praesent et fermentum ante, nec molestie felis. Integer imperdiet mauris eget ipsum laoreet, quis blandit orci tempor. Etiam tristique in urna ut luctus. Nullam et arcu nec lacus dignissim tincidunt. Fusce dapibus, felis sit amet lacinia suscipit, nisl neque fringilla est, sed consectetur turpis augue eu elit. Nam scelerisque, tortor vel cursus mollis, urna dolor dignissim eros, in euismod arcu est sit amet elit.

Pellentesque lacinia leo ex, luctus vestibulum purus dapibus varius. In purus justo, egestas porttitor facilisis eget, pretium hendrerit eros. Sed dignissim vulputate tortor, ut consequat dui malesuada id. Aliquam mattis odio sed libero congue lobortis. Suspendisse potenti. Morbi sit amet ultrices nulla. Phasellus metus dolor, mollis ut diam non, sollicitudin efficitur elit.

Etiam condimentum rutrum placerat. Sed pulvinar sollicitudin suscipit. Aenean facilisis hendrerit lorem. Etiam vulputate accumsan est ac ullamcorper. Proin eu cursus leo. Fusce metus lacus, ullamcorper a pellentesque at, consectetur eget est. Etiam malesuada pretium velit. Duis dignissim justo vitae mauris fermentum vehicula. Proin lobortis urna dui. Sed id dui odio. Duis quis porta justo, ac tristique nisi.

Cras ullamcorper ultricies mi, vel fringilla lectus malesuada sit amet. Fusce id elit non quam faucibus vulputate. Vestibulum tortor lorem, fermentum nec sagittis eu, lacinia sed mauris. Curabitur pellentesque purus sit amet vestibulum elementum. Mauris a neque nec odio sagittis auctor. Nam quis volutpat dui, id blandit magna. Integer aliquam bibendum velit ac auctor.

Nulla consectetur convallis leo, ac elementum ante placerat a. Morbi nec imperdiet nisi, eu rutrum est. Quisque posuere, ligula ac sagittis viverra, neque nibh condimentum metus, quis placerat nisl enim a neque. Nam consectetur ligula vel eros euismod pharetra. Morbi tempus enim et hendrerit fermentum. Pellentesque at posuere odio. Etiam nec nisl quis urna volutpat scelerisque. Aenean convallis leo mauris, eget commodo libero dignissim sagittis. Aenean vitae elit interdum, tincidunt quam at, rutrum est. Nunc quis imperdiet ligula. Maecenas libero lectus, volutpat eget lobortis ac, sagittis vitae sapien. Sed non molestie nulla. Praesent cursus, augue at ultricies interdum, velit nibh sagittis metus, interdum cursus dui libero vel leo.

Phasellus vel eros leo. Suspendisse aliquam purus in commodo congue. Suspendisse nec tristique leo. Ut ut nulla turpis. Orci varius natoque penatibus et magnis dis parturient montes, nascetur ridiculus mus. Donec condimentum risus quis erat maximus finibus. Pellentesque blandit iaculis est, a molestie urna mattis eu. Donec ullamcorper faucibus enim scelerisque tincidunt. Maecenas condimentum sodales elit, sit amet vulputate ligula dapibus id. Sed laoreet rhoncus urna vel aliquam. Phasellus auctor elementum lectus, et auctor dolor fringilla vitae. Ut nec tellus non ipsum efficitur vestibulum porttitor vel urna. Vestibulum vulputate justo dolor, vitae facilisis eros eleifend mollis.

Sed ut est molestie neque fringilla egestas vel ac leo. Donec tempus viverra libero, sit amet pellentesque massa tincidunt molestie. Praesent eget commodo metus. Mauris semper tristique leo, quis volutpat arcu pellentesque eget. Curabitur in quam aliquam, pellentesque turpis vel, mollis felis. Suspendisse non diam eget velit semper varius ut at tellus. Donec nibh est, elementum ac fermentum a, dictum et lacus. Cras luctus, ante id sagittis interdum, libero turpis pulvinar augue, eu pharetra nisi odio et odio. Nunc sodales suscipit libero, imperdiet elementum ante aliquam a. Quisque quis risus dui. Pellentesque habitant morbi tristique senectus et netus et malesuada fames ac turpis egestas. Vestibulum varius, orci eget feugiat ultricies, lectus est vulputate lectus, aliquet egestas augue lacus eu mi. Aliquam erat volutpat. Etiam rhoncus ex at quam pretium sagittis. Proin nec placerat nibh, vitae luctus quam.

\chapter{Lý Thuyết Về Thị Giác Máy Tính}
\label{computer_vision_background}

The quick brown fox jumps over the lazy dog

% Hiển thị code từ file
\lstinputlisting[language=C++]{SourceCode/hello.cpp}

\begin{lstlisting}[language=Python]
print("Hello, World!!!")
\end{lstlisting}

\begin{lstlisting}[language=JavaScript]
console.log('Hello, World!!!');
\end{lstlisting}

% Sử dụng block \begin{figure} để LaTex tự động thêm reference cho ảnh đến trang danh sách hình ảnh.
% [h]: tham khảo thêm tại https://www.overleaf.com/learn/latex/Inserting_Images#Placement
\begin{figure}[h]
	\centering
	\includegraphics[scale=0.75]{logo-khtn}
	\caption{Logo của Trường Đại học Khoa Học Tự Nhiên}
\end{figure}

% Sử dụng block \begin{table} để LaTex tự động thêm reference cho bảng đến trang danh sách bảng.
\begin{table}[h!]
	\centering
	\begin{tabular}{||c c c c||} 
		\hline
		Col1 & Col2 & Col2 & Col3 \\ [0.5ex] 
		\hline\hline
		1 & 6 & 87837 & 787 \\ 
		2 & 7 & 78 & 5415 \\
		3 & 545 & 778 & 7507 \\
		4 & 545 & 18744 & 7560 \\
		5 & 88 & 788 & 6344 \\ [1ex] 
		\hline
	\end{tabular}
	\caption{Table to test captions and labels}
	\label{table:1}
\end{table}

Tham khảo mẫu: \cite{1984-TeX-Knuth}

% Appendix - chứa trang chú thích và tài liệu tham khảo

% In tài liệu tham khảo

\phantomsection
\addcontentsline{toc}{chapter}{\mdseries Chú thích}
\include{Appendix/appendix}

\DeclareNameAlias{sortname}{last-first}
\DeclareNameAlias{default}{last-first}

\phantomsection
\addcontentsline{toc}{chapter}{\mdseries Tham khảo}
\printbibliography[title=Tham khảo]

% Phần phụ lục
%\include{Appendix/appendix1}
%\include{Appendix/appendix2}

\end{document} 